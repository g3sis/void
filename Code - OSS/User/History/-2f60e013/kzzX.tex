\documentclass{beamer}
\usepackage[utf8]{inputenc}
\usepackage{helvet}
\usepackage[english, ngerman]{babel}

\usecolortheme{tum}
\useoutertheme{tum}

\setbeamerfont{author}{size=\footnotesize}
\setbeamerfont{date}{size=\scriptsize}
\setbeamerfont{date}{size=\scriptsize}

\useinnertheme{rectangles}

\usepackage{pgf}  
\usepackage{tikz}
\logo{\pgfputat{\pgfxy(-0.2, 8.9)}{\pgfbox[right,top]{
\begin{tikzpicture}[y=0.38pt, x=0.38pt,yscale=-1, inner sep=0pt, outer sep=0pt]
\begin{scope}[cm={{1.25,0.0,0.0,-1.25,(0.0,35.4325)}}]
    \path[fill=tum,nonzero rule] (4.8090,23.2950) -- (4.8090,-0.0020) --
      (9.8590,-0.0020) -- (9.8590,23.2600) -- (15.4730,23.2600) -- (15.4730,-0.0020)
      -- (31.5390,-0.0020) -- (31.5390,23.0140) -- (37.2580,23.0140) --
      (37.2580,0.0060) -- (42.5550,0.0060) -- (42.5550,23.0140) -- (48.3440,23.0140)
      -- (48.3440,0.0060) -- (53.6410,0.0060) -- (53.6410,28.3460) --
      (26.4530,28.3460) -- (26.4530,5.1580) -- (20.6290,5.1580) -- (20.6290,28.3110)
      -- (-0.0000,28.3110) -- (-0.0000,23.2950) -- (4.8090,23.2950) -- cycle;
\end{scope}
\end{tikzpicture}
}}}

\setbeamertemplate{title page}
{
	\vbox{}
	\vfill
	\begin{flushleft}
		\begin{beamercolorbox}[sep=8pt,left]{title}
			\usebeamerfont{title}\inserttitle\par%
			\ifx\insertsubtitle\@empty%
			\else%
				\vskip0.25em%
				{\usebeamerfont{subtitle}\usebeamercolor[fg]{subtitle}\insertsubtitle\par}%
			\fi%
    	\end{beamercolorbox}%
    	\vskip1em\par
		\begin{beamercolorbox}[sep=8pt,left]{author}
		\usebeamerfont{author}\insertauthor
		\end{beamercolorbox}
		\begin{beamercolorbox}[sep=8pt,left]{institute}
		\usebeamerfont{institute}\insertinstitute
		\end{beamercolorbox}
		\begin{beamercolorbox}[sep=8pt,left]{date}
		\usebeamerfont{date}\insertdate
		\end{beamercolorbox}\vskip0.5em
		{\usebeamercolor[fg]{titlegraphic}\inserttitlegraphic\par}
	\end{flushleft}
	\vfill
}

\mode<presentation>

\title{A collection of In-Band Trust Establishment Techniques}

\author{Robin Möller}
\institute[]{Technical University Munich}

\begin{document}

\begin{frame}
\titlepage
\end{frame}

\begin{frame}
\frametitle{Structure}
\begin{itemize}
	\item Background
	\item Different Techniques
	\item Evaluation
\end{itemize}
\end{frame}

\begin{frame}
	\frametitle{Background}
	\framesubtitle{Essentials}
	\begin{itemize}
		\item Manchester Codes
		\item On-Off keying
		\item Different Attacks
	\end{itemize}
\end{frame}

\begin{frame}
	\frametitle{Background}
	\framesubtitle{Attacks}
	\begin{itemize}
		\item MITM
		\item Signal cancelation
		\item Signal overshadowing
	\end{itemize}
\end{frame}

\begin{frame}
	\frametitle{Different Techniques}
	\framesubtitle{Overview}
	\begin{itemize}
		\item Integrity Codes
		\item Tamper Evident Pairing
		\item Chorus
		\item Help
		\item VERSE
		\item SFire
	\end{itemize}
\end{frame}

\begin{frame}
	\frametitle{Integrity Codes}
	\begin{itemize}
		\item 
		\item BUT: Attacker can hide that message has been transmitted
	\end{itemize}
\end{frame}

\begin{frame}
	\frametitle{Integrity Codes}
	\framesubtitle{Structure}
	\includegraphics[width=\textwidth]{img/ICodes.png}
\end{frame}

\begin{frame}
	\frametitle{Tamper-Evident Pairing}
	\framesubtitle{}
	\begin{itemize}
		\item First wireless in-band pairing protocol
		\item \textit{Tamper-Evident Announcement} primitive
	\end{itemize}
\end{frame}

\begin{frame}
	\frametitle{Tamper-Evident Pairing}
	\framesubtitle{Tamper-Evident Announcement}
	\begin{figure}
		\centering
		\includegraphics[width=\textwidth]{img/TEA.png}
		\caption{format of TEA}
		\label{fig:tea}
	\end{figure}
\end{frame}

\begin{frame}
	\frametitle{Chorus}
	\framesubtitle{}
	\begin{itemize}
		\item Partialy inspired by I-Codes and TEP
		\item But much more scalable 
	\end{itemize}
\end{frame}

\begin{frame}
	\frametitle{Chorus}
	\framesubtitle{Authenticated Equality Comparison}
	\begin{itemize}
		\item Non-spoofing 
		\item Correctness
		\item Non-blocking
	\end{itemize}
\end{frame}

\begin{frame}
	\frametitle{Chorus}
	\framesubtitle{Structure}
	\begin{itemize}
		\item Synchronization packet 
		\item CTS\_TO\_SELF
		\item Comparison Phase
	\end{itemize}
\end{frame}

\begin{frame}
	\frametitle{Help}
	\framesubtitle{}
	\begin{itemize}
		\item 
	\end{itemize}
\end{frame}

\begin{frame}
	\frametitle{VERSE}
	\framesubtitle{}
	\begin{itemize}
		\item 
	\end{itemize}
\end{frame}

\begin{frame}
	\frametitle{SFire}
	\framesubtitle{}
	\begin{itemize}
		\item 
	\end{itemize}
\end{frame}

\begin{frame}
	\frametitle{Evaluation}
	\framesubtitle{}
	\begin{itemize}
		\item 
	\end{itemize}
\end{frame}

\end{document}