\documentclass{beamer}
\usepackage[utf8]{inputenc}
\usepackage{helvet}
\usepackage[english, ngerman]{babel}

\usecolortheme{tum}
\useoutertheme{tum}

\setbeamerfont{author}{size=\footnotesize}
\setbeamerfont{date}{size=\scriptsize}
\setbeamerfont{date}{size=\scriptsize}

\useinnertheme{rectangles}

\usepackage{pgf}  
\usepackage{tikz}
\logo{\pgfputat{\pgfxy(-0.2, 8.9)}{\pgfbox[right,top]{
\begin{tikzpicture}[y=0.38pt, x=0.38pt,yscale=-1, inner sep=0pt, outer sep=0pt]
\begin{scope}[cm={{1.25,0.0,0.0,-1.25,(0.0,35.4325)}}]
    \path[fill=tum,nonzero rule] (4.8090,23.2950) -- (4.8090,-0.0020) --
      (9.8590,-0.0020) -- (9.8590,23.2600) -- (15.4730,23.2600) -- (15.4730,-0.0020)
      -- (31.5390,-0.0020) -- (31.5390,23.0140) -- (37.2580,23.0140) --
      (37.2580,0.0060) -- (42.5550,0.0060) -- (42.5550,23.0140) -- (48.3440,23.0140)
      -- (48.3440,0.0060) -- (53.6410,0.0060) -- (53.6410,28.3460) --
      (26.4530,28.3460) -- (26.4530,5.1580) -- (20.6290,5.1580) -- (20.6290,28.3110)
      -- (-0.0000,28.3110) -- (-0.0000,23.2950) -- (4.8090,23.2950) -- cycle;
\end{scope}
\end{tikzpicture}
}}}

\setbeamertemplate{title page}
{
	\vbox{}
	\vfill
	\begin{flushleft}
		\begin{beamercolorbox}[sep=8pt,left]{title}
			\usebeamerfont{title}\inserttitle\par%
			\ifx\insertsubtitle\@empty%
			\else%
				\vskip0.25em%
				{\usebeamerfont{subtitle}\usebeamercolor[fg]{subtitle}\insertsubtitle\par}%
			\fi%
    	\end{beamercolorbox}%
    	\vskip1em\par
		\begin{beamercolorbox}[sep=8pt,left]{author}
		\usebeamerfont{author}\insertauthor
		\end{beamercolorbox}
		\begin{beamercolorbox}[sep=8pt,left]{institute}
		\usebeamerfont{institute}\insertinstitute
		\end{beamercolorbox}
		\begin{beamercolorbox}[sep=8pt,left]{date}
		\usebeamerfont{date}\insertdate
		\end{beamercolorbox}\vskip0.5em
		{\usebeamercolor[fg]{titlegraphic}\inserttitlegraphic\par}
	\end{flushleft}
	\vfill
}

\mode<presentation>

\title{A collection of In-Band Trust Establishment Techniques}

\author{Robin Möller}
\institute[]{Technical University Munich}
\date[01.02.2024]{01.Februar 2024}
\begin{document}

\begin{frame}
\titlepage
\end{frame}

\begin{frame}
\frametitle{Structure}
\begin{itemize}
	\item Background
	\item Different Techniques
	\item Evaluation
\end{itemize}
\end{frame}

\begin{frame}
	\frametitle{Background}
	\framesubtitle{Essentials}
	\begin{itemize}
		\item Manchester Codes
		\item On-Off keying
		\item Different Attacks
	\end{itemize}
\end{frame}

\begin{frame}
	\frametitle{Background}
	\framesubtitle{Attacks}
	\begin{itemize}
		\item Man in the middle
		\item Signal cancelation
		\item Signal overshadowing
	\end{itemize}
\end{frame}

\begin{frame}
	\frametitle{Different Techniques}
	\framesubtitle{Overview}
	\begin{itemize}
		\item Integrity Codes
		\item Tamper Evident Pairing
		\item Chorus
		\item Help
		\item VERSE
	\end{itemize}
\end{frame}

\begin{frame}
	\frametitle{Integrity Codes}
	\framesubtitle{Structure}
	\begin{itemize}
		\item Finite set of Source states $\mathcal{S}$
		\item Finite set of codewords $\mathcal{C}$
		\item source encoding rule $e: \mathcal{S}\rightarrow\mathcal{C}$
	\end{itemize}
\end{frame}

\begin{frame}
	\frametitle{Integrity Codes}
	\framesubtitle{Delimiters}
	\begin{itemize}
		\item Used for synchronization
		\item Cannot be converted into a codeword without flipping a ''1'' bit
		\item A Codeword cannot be converted into a delimiter without flipping a ''1'' bit 
		\item Every valid codeword between to delimiters is authentic  
	\end{itemize}
\end{frame}
  
\begin{frame}
	\frametitle{Integrity Codes}
	\framesubtitle{Structure}
	\includegraphics[width=\textwidth]{img/ICodes.png}
\end{frame}

\begin{frame}
	\frametitle{Integrity Codes}
	\framesubtitle{Authenticaton through presence}
	\begin{itemize}
		\item Condition of presence
		\item Condition of synchronization
		\item Can be used in access point authentication or in combination with Diffie-Hellman 
	\end{itemize}
\end{frame}

\begin{frame}
	\frametitle{Tamper-Evident Pairing}
	\framesubtitle{}
	\begin{itemize}
		\item First wireless in-band pairing protocol
		\item Based on push button configuration
		\item \textit{Tamper-Evident Announcement} primitive
	\end{itemize}
\end{frame}

\begin{frame}
	\frametitle{Tamper-Evident Pairing}
	\framesubtitle{Protocol}
	\begin{itemize}
		\item Enrollee enters PBC mode and sends TEA message
		\item Registrar responds with TEA message 
	\end{itemize}
\end{frame}

\begin{frame}
	\frametitle{Tamper-Evident Pairing}
	\framesubtitle{Tamper-Evident Announcement}
	\begin{figure}
		\centering
		\includegraphics[width=\textwidth]{img/TEA.png}
	\end{figure}
\end{frame}

\begin{frame}
	\frametitle{Chorus}
	\framesubtitle{}
	\begin{itemize}
		\item Partialy inspired by I-Codes and TEP
		\item But much more scalable
		\item Used in message authentication protocols 
	\end{itemize}
\end{frame}

\begin{frame}
	\frametitle{Chorus}
	\framesubtitle{Authenticated Equality Comparison}
	\begin{itemize}
		\item Non-spoofing 
		\item Correctness
		\item Non-blocking
	\end{itemize}
\end{frame}

\begin{frame}
	\frametitle{Chorus}
	\framesubtitle{Structure}
	\begin{itemize}
		\item Synchronization packet 
		\item CTS\_TO\_SELF
		\item Comparison Phase
	\end{itemize}
\end{frame}

\begin{frame}
	\frametitle{Chorus}
	\framesubtitle{Comparison Phase}
	\begin{itemize}
		\item Each node encodes and sends a bitstring 
		\item Detection of noise during an OFF-Slot leads to rejection 
	\end{itemize}
\end{frame}

%%\begin{frame}
%%	\frametitle{Chorus}
%%	\framesubtitle{Security of basic Chorus}
%%	\begin{itemize}
%%		\item Signal during ON-Slots is random
%%		\item Flipping OFF to ON causes abort at all nodes
%%		\item Enhanced Chorus against relay attacks 
%%	\end{itemize}
%%\end{frame}

\begin{frame}
	\frametitle{Chorus}
	\framesubtitle{FH-Corus}
	\begin{itemize}
		\item Uses Uncoordinated Frequency Hopping
		\item Extension of each slot to minislots 
	\end{itemize}
\end{frame}

\begin{frame}
	\frametitle{Chorus}
	\framesubtitle{Protocols}
	\begin{itemize}
		\item In-band MAP with short authentication string
		\item In-band MAP with long authentication string
		\item Both used in group MAPs
	\end{itemize}
\end{frame}

\begin{frame}
	\frametitle{Help}
	\framesubtitle{Basic Setup}
	\begin{figure}
		\includegraphics[width=0.8\textwidth]{img/HELP.png}
	\end{figure}
\end{frame}

\begin{frame}
	\frametitle{Help}
	\framesubtitle{Integrity protection}
	\begin{itemize}
		\item Helper device in close proximity to pairing device
		\item Helper sends signal with random ON-Slots synchronously to the normal devices message
		\item Helper reveals the ON-Slots to the base station
		\item Base station verifies both messagest
	\end{itemize}
\end{frame}

\begin{frame}
	\frametitle{Help}
	\framesubtitle{Signal Example}
	\begin{figure}
		\includegraphics[width=0.8\textwidth]{img/HELP2.png}
	\end{figure}
\end{frame}

\begin{frame}
	\frametitle{Help}
	\framesubtitle{Usage}
	\begin{itemize}
		\item Diffie-Hellman
	\end{itemize}
\end{frame}

\begin{frame}
	\frametitle{VERSE}
	\framesubtitle{Structure}
	\begin{figure}
		\includegraphics[width=0.8\textwidth]{img/Verse.png}
	\end{figure}

\end{frame}

\begin{frame}
	\frametitle{VERSE}
	\framesubtitle{PHY-Layer primitive}
	\begin{itemize}
		\item Compilation of protocol transcript
		\item Synchronization
		\item Transcript verification
	\end{itemize}
\end{frame}

\begin{frame}
	\frametitle{VERSE}
	\framesubtitle{Protocol}
	\begin{itemize}
		\item Combination of phy-layer primitive and DH	
	\end{itemize}
\end{frame}

\begin{frame}
	\frametitle{}
	\framesubtitle{}
	\begin{center}
		Thank your for your attention!\\
		Questions?
	\end{center}
\end{frame}

\end{document}