\documentclass[conference]{IEEEtran}
\IEEEoverridecommandlockouts
% The preceding line is only needed to identify funding in the first footnote. If that is unneeded, please comment it out.
\usepackage{cite}
\usepackage{amsmath,amssymb,amsfonts}
\usepackage{algorithmic}
\usepackage{graphicx}
\usepackage{textcomp}
\usepackage{xcolor}
\def\BibTeX{{\rm B\kern-.05em{\sc i\kern-.025em b}\kern-.08em
    T\kern-.1667em\lower.7ex\hbox{E}\kern-.125emX}}
\begin{document}

\title{A collection of In-Band Trust Establishment Techniques\\
}

\author{\IEEEauthorblockN{Möller, Robin}
\IEEEauthorblockA{
\textit{Technical University Munich}\\
Munich, Germany \\
robin.moeller@tum.de}
}

\maketitle

\begin{abstract}

\end{abstract}

\begin{IEEEkeywords}
component, formatting, style, styling, insert
\end{IEEEkeywords}

\section{Introduction}

\section{Background}
\begin{itemize}
    \item on-ff keying
    \item attacker models
    \item poepper et al. signal cancelation attack
\end{itemize}

\subsection{Integrity Codes}

Codes that enable the receiver of a message to detect integrity violations during the transmission are genarally known as \textit{All-Unidirectional Error-Detecting} Codes.
They are used in settings, where a bit ''0'' can be changed into a bit ''1'' but the reverse transformation is not feasible. 
Cagalj et al provided the idea of integrity codes (I-codes), where a recipient can ensure the integrity of a message based on the encoding.
I-Codes consist of a tripple ($ \mathcal{S}, \mathcal{C}, e$), where $\mathcal{S}$ is a finite set of possible source states, $\mathcal{C}$ resembles a set of codewords that is also finite and $e$ is a source encoding rule with two limitations.
The first being that $e$ has to be an injective function and the second being the impossibility of converting a codeword $c$ to another codeword $c'$ where $c \neq c'$ without flipping at least one bit ''1''.
This mechanic is used in combination with on-off keying and signal antiblocking to ensure the detectibility of attacks on the messages integrity. We will call the whole construction of these techniques I-codes for simplicity reasons.
These attacks could be executed by an adversary in a man in the middle setting by trying to flip bits using several signal modulation techniques or simply signal overshadowing.
To ensure that ''1'' bits cannot be turned into ''0'' bits and hereby canceld out, the symbol is transmitted as a random signal, which the attacker would have to predict the exact shape of in order to erase it.
The reverse direction of bit flipping is possible, but given the use of on-off key modulation, the attacker must erase at least one signal to preserve the message's validity. Any invalid message is simply dropped by the receiver.
The complementary encoding used for on-off keying makes synchronization between sender and receiver mandatory. It is achieved through the use of specially designed bit strings known as ``delimiters''. The sender transmits a signal containing a sequence of the message bracketed by two consecutive delimiters. Simultaneously, the receiver is actively monitoring the channel for any transmitted messages. 
Upon detecting delimiters, the receiver proceeds to decode the enclosed message. If the message can be decoded using the inverse of the I-codes function $e$, it is accepted. Delimiters are defined by the following rules: Firstly, no substring of any given codeword can be transformed into a valid delimiter without flipping at least one bit from ''1'' to ''0''. Secondly, the delimiter cannot be transformed into any substring of any codeword by the same process. Lastly, any codeword received between two delimiters is considered valid.
During experimentation with these concepts, Capkun et al. discovered that I-Codes offer sufficient robustness for the transmission of short messages. Additionally, they developed a new concept called ``Authentication through presence'' and outlined two exemplary use cases.
The protocol works by sending the message $m$ whose integrity needs to be proven over a channel $C_1$ and a message digest $h(m)$ is sent over another channel $C_2$ using I-Coding with $h(\dot)$ being a one way function used to protect the integrity of the  message.

TODO
- novel concept "Authentication through presence"
    access points
    key establishment over insecure channel
        DH-KeyAgreement

- security analysis


\section{Tamper-Evident Pairing}

\section{Chorus}

Chorus is a technique for initializing in-band trust. It was created by Hou et al. as a physical layer primitive, offering scalability.
Its design draws significant inspiration from Tamper-Evident Pairing and I-Codes, and it is utilized to build secure group key agreement protocols.
The key idea behind The Chorus is that each device sends a specially encoded authentication string to every other connected device synchronously. 
Any discrepancy in these strings will be detected by all devices, each of which outputs only a single bit of information - it either accepts (1) or rejects (0).
Given the unidirectional property of the wireless channel, the likelihood of altering a positive result is negligible. The Chorus does not directly authenticate a sent message. Instead, a method introduced by Hou et al. called ``Authenticated Equality Comparison'' or short AEC is used to achive authentication of N bit strings, which are derived from the original message. 
AEC requires 3 key characteristics. It has to be \textit{Non-Spoofing}, which means that if messages $i$ and $j$ exist and the corresponding binary strings $s_i$ and $s_j$ are not the same, the outputs for every $i \in (1, . . . , N)$ are rejected with a high probabiliy. The second property that has to be met is \textit{Correctnes}. If for every pair messages $i,j \in (1, . . . , N)$ the corresponding binary strings $s_i$ and $s_j$ are equal, the criterium is met. The third atribbute is \textit{Non-Blocking} which just means that AEC cannot be blocked from happening, delayed or hidden.
The basic Chorus starts if one node sends a synchronization packet which is longer than regular packets and contains random information. This node will be refered to as the coordinator. The existence of such a package is noticed by all other nodes via threashold energy detection. Afterwards the coordinator waits for a short period of time and reserves the channel for the time until Chorus deliveres a result by sending a CTS\textbackslash \_TO\textbackslash \_SELF packet.During the third phase each node \$i\$ sends a manchester encoded bit string using on-off keying simultaneously. During ON-Slots random information is transmited and during OFF-Slots the nodes keep silent and simply listen to the channel. If a node detects energy during an OFF-Slot, it outputs reject, otherwise it outputs accept.
This realization is secure against a type-I signal cancellation attacker, because the attacking party would have to cancel out the sum of the random noise sent by each node during each `on' slot. Simply flipping `0' bits to `1' bits would just lead to the sending of abort messages. Self-cancellation does not significantly affect Chorus. \\
\texttt{TODO: TYPE 2 attacker} \\

\section{HELP}

\section{SFire}

\section{VERSE}

\section{Other Methods}

\section{Evaluation}

\section{Discussion}

\section{Related Work}

\section{Conclusion}

\section*{Acknowledgment}

\section*{References}


\begin{thebibliography}{00}
\bibitem{b1} G. Eason, B. Noble, and I. N. Sneddon, ``On certain integrals of Lipschitz-Hankel type involving products of Bessel functions,'' Phil. Trans. Roy. Soc. London, vol. A247, pp. 529--551, April 1955.
\bibitem{b2} J. Clerk Maxwell, A Treatise on Electricity and Magnetism, 3rd ed., vol. 2. Oxford: Clarendon, 1892, pp.68--73.
\bibitem{b3} I. S. Jacobs and C. P. Bean, ``Fine particles, thin films and exchange anisotropy,'' in Magnetism, vol. III, G. T. Rado and H. Suhl, Eds. New York: Academic, 1963, pp. 271--350.
\bibitem{b4} K. Elissa, ``Title of paper if known,'' unpublished.
\bibitem{b5} R. Nicole, ``Title of paper with only first word capitalized,'' J. Name Stand. Abbrev., in press.
\bibitem{b6} Y. Yorozu, M. Hirano, K. Oka, and Y. Tagawa, ``Electron spectroscopy studies on magneto-optical media and plastic substrate interface,'' IEEE Transl. J. Magn. Japan, vol. 2, pp. 740--741, August 1987 [Digests 9th Annual Conf. Magnetics Japan, p. 301, 1982].
\bibitem{b7} M. Young, The Technical Writer's Handbook. Mill Valley, CA: University Science, 1989.
\end{thebibliography}

\end{document}
